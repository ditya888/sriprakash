\documentclass[12pt]{article}
\usepackage{amsmath}
\newcommand{\mydet}[1]{\ensuremath{\begin{vmatrix}#1\end{vmatrix}}}
\newcommand{\myvec}[1]{\ensuremath{\begin{pmatrix}#1\end{pmatrix}}}
\newcommand{\solution}{\noindent \textbf{Solution: }}
\providecommand{\brak}[1]{\ensuremath{\left(#1\right)}}
\providecommand{\norm}[1]{\left\lVert#1\right\rVert}
\let\vec\mathbf

\title{Coordinate Geometry}
\author{adityatanish.chakka@sriprakashschools.com}

\begin{document}
\maketitle
\section*{Class 10$^{th}$ Maths - Chapter 7}
This is Problem-10 from Exercise 7.2
\begin{enumerate}
\item Find the area of the rhonmbus whose vertices are:
(3,0),(4,5),(-1,4),(-2,-1)
\begin{align}
{(3,0),(4,5),(-1,4),(-2,-1)}
\end{align}
\solution \\
Given Data:
\begin{align}
A = \myvec{3\\0}\\
B = \myvec{-4\\5}\\
C = \myvec{-1\\-4}\\
D = \myvec{-2\\-1}\\
\end{align}
\begin{align}
\vec{AC} = \myvec{4\\0}\\
\vec{BD} = \myvec{6\\6}\\
\end{align}
AREA OF A RHOMBUS;
\begin{align}
\frac{1}{2}\norm{\vec{A}-\vec{C}\times \vec{B}-\vec{D}}\\
\frac{1}{2}\mydet{-4 & 0\\ -6 & -6}\\
\frac{1}{2}\norm{24 + 0}\\
\frac{1}{2}\norm{24}\\
\frac{12}{1} sq.units\\
12 sq.units\\
therefore the area of the given rhombus is 12 sq.units
\end{align}



    

    


\
\end{enumerate}
\end{document}